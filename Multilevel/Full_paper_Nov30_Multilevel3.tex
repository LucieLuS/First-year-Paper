\documentclass[12pt,]{article}
\usepackage{lmodern}
\usepackage{amssymb,amsmath}
\usepackage{ifxetex,ifluatex}
\usepackage{fixltx2e} % provides \textsubscript
\ifnum 0\ifxetex 1\fi\ifluatex 1\fi=0 % if pdftex
  \usepackage[T1]{fontenc}
  \usepackage[utf8]{inputenc}
\else % if luatex or xelatex
  \ifxetex
    \usepackage{mathspec}
  \else
    \usepackage{fontspec}
  \fi
  \defaultfontfeatures{Ligatures=TeX,Scale=MatchLowercase}
\fi
% use upquote if available, for straight quotes in verbatim environments
\IfFileExists{upquote.sty}{\usepackage{upquote}}{}
% use microtype if available
\IfFileExists{microtype.sty}{%
\usepackage{microtype}
\UseMicrotypeSet[protrusion]{basicmath} % disable protrusion for tt fonts
}{}
\usepackage[margin = 1.2in]{geometry}
\usepackage{hyperref}
\hypersetup{unicode=true,
            pdftitle={Research Note: Why and When do Leaders of Developing Countries Negotiate South-North Preferential Trade Agreements?},
            pdfauthor={Lucie Lu},
            pdfborder={0 0 0},
            breaklinks=true}
\urlstyle{same}  % don't use monospace font for urls
\usepackage{graphicx,grffile}
\makeatletter
\def\maxwidth{\ifdim\Gin@nat@width>\linewidth\linewidth\else\Gin@nat@width\fi}
\def\maxheight{\ifdim\Gin@nat@height>\textheight\textheight\else\Gin@nat@height\fi}
\makeatother
% Scale images if necessary, so that they will not overflow the page
% margins by default, and it is still possible to overwrite the defaults
% using explicit options in \includegraphics[width, height, ...]{}
\setkeys{Gin}{width=\maxwidth,height=\maxheight,keepaspectratio}
\IfFileExists{parskip.sty}{%
\usepackage{parskip}
}{% else
\setlength{\parindent}{0pt}
\setlength{\parskip}{6pt plus 2pt minus 1pt}
}
\setlength{\emergencystretch}{3em}  % prevent overfull lines
\providecommand{\tightlist}{%
  \setlength{\itemsep}{0pt}\setlength{\parskip}{0pt}}
\setcounter{secnumdepth}{0}
% Redefines (sub)paragraphs to behave more like sections
\ifx\paragraph\undefined\else
\let\oldparagraph\paragraph
\renewcommand{\paragraph}[1]{\oldparagraph{#1}\mbox{}}
\fi
\ifx\subparagraph\undefined\else
\let\oldsubparagraph\subparagraph
\renewcommand{\subparagraph}[1]{\oldsubparagraph{#1}\mbox{}}
\fi

%%% Use protect on footnotes to avoid problems with footnotes in titles
\let\rmarkdownfootnote\footnote%
\def\footnote{\protect\rmarkdownfootnote}

%%% Change title format to be more compact
\usepackage{titling}

% Create subtitle command for use in maketitle
\newcommand{\subtitle}[1]{
  \posttitle{
    \begin{center}\large#1\end{center}
    }
}

\setlength{\droptitle}{-2em}

  \title{Research Note: Why and When do Leaders of Developing Countries Negotiate
South-North Preferential Trade Agreements?}
    \pretitle{\vspace{\droptitle}\centering\huge}
  \posttitle{\par}
    \author{Lucie Lu}
    \preauthor{\centering\large\emph}
  \postauthor{\par}
      \predate{\centering\large\emph}
  \postdate{\par}
    \date{October 20, 2018}

\usepackage{booktabs}
\usepackage{longtable}
\usepackage{array}
\usepackage{multirow}
\usepackage[table]{xcolor}
\usepackage{wrapfig}
\usepackage{float}
\usepackage{colortbl}
\usepackage{pdflscape}
\usepackage{tabu}
\usepackage{threeparttable}
\usepackage{threeparttablex}
\usepackage[normalem]{ulem}
\usepackage{makecell}

\usepackage{placeins}
\usepackage{setspace}
\usepackage{chngcntr}
\usepackage{pdflscape}
\onehalfspacing
\counterwithin{figure}{section}
\counterwithin{table}{section}
\usepackage{float}
\floatplacement{figure}{H}

\begin{document}
\maketitle

\(\qquad\) Why and when will leaders in the developing countries
negotiate South-North preferential trade agreements with provisions of
economic reforms? This paper is interested in explaining what motivates
a leader from the developing countries to negotiate this particular deep
and demanding international treaty. Preferential trade agreements
(hereafter PTAs) are binding international treaties that help states to
foster trade and economic integration among member-states. Current
literature on the preferential trade agreement mainly focuses on the
effects of this institution. Questions centre around whether the PTAs
increase trade volumes among member-states and their impacts on the
overall trade flows in the multilateral trade system (Bagwell and
Staiger 1998). Put simply, scholars examine to what extent the PTAs have
achieved their desired economic purposes: lowering tariffs, facilitating
trade and increasing the welfare of the states in the long term. In
addition to studying the consequences of the PTAs, emerging political
economy literature has offered explanations for why states form the
PTAs. Earlier years, the focus is the macroeconomic and macropolitical
factors that explain the proliferation of the PTAs, including the lack
of the progress in the multilateral trade regime and the spillover
effect of regionalism (Baldwin 2012; Rodrik 1992; Mansfield and
Reinhardt 2003).

\(\qquad\) One trend of the current literature on the PTA focuses on the
effects, others on the cause, which is the theme that this paper speaks
to. Recent studies zoom into the domestic factors within the states to
explain this trend. While there is no shortage of explanations on why
governments form PTAs, scholars also tend to treat the PTAs as a unitary
concept (Dür, Baccini, and Elsig 2014). They overlook the design
differences of the PTAs, which imply different degrees of policy changes
embedded in these international agreements. Few studies reach beyond
trade liberalization to examine the role of the PTA in policy reform.
Perhaps the most important exception is Baccini and Urpelainen (2014),
who argue PTA can help developing countries to implement economic
reforms, leaning more on examining the effect of the PTA. Picking up on
this approach, this article argues the leaders' commitments to such
costly agreement reflect their political purposes. Leaders have
political objectives to commit to the deep PTAs with the expectation to
conduct economic reforms after they face political challenge. The
political motivation of PTA negotiation is currently understudied; yet,
to understand when and why a leader of developing countries commits to
the preferential trade agreement with the provision of economic reform
is important because economic reforms will have immense consequences for
the citizens of that country and beyond. The key question, the cause of
economic reforms, yet, remains open.

\(\qquad\) The puzzle to explain in this paper is what motivates a
leader in the developing country to negotiate one of the most costly and
stringent trade agreements with the developed country. One of the most
important features of these PTAs is the provision of economic reforms,
whose implementation will ignite domestic opposition. My argument, in
brief, is the following: a leader strategically negotiates a PTA when he
is insecure. I argue after a leader is hit by a shock to security, he
can choose economic reforms to punish the disloyal opposition
responsible for the outbreak. Leaders are more likely to implement
economic reform when the PTAs negotiation is in place. The PTAs with
major economic powers send positive signals to mitigate commitment
problem that a leader has, and offers material benefits to facilitate
the reforms. This argument has a main empirical implication that can be
used to test its validity in the next section. My prediction is:
Negative shocks to a leader's security increase the probability of PTA
negotiations.

\section{Research Design}\label{research-design}

\(\qquad\) My research design compares the likelihoods of leaders with
shocks to security and those without shocks to negotiate the South-North
PTAs. The hypothetical treatment in this study is \emph{shock to
security}.

\(\qquad\) I defined two types of shocks to security. Before getting
into a shock, here I use two indices to measure the security of a
leader:

\begin{enumerate}
\def\labelenumi{\arabic{enumi})}
\item
  a leader's security before he starts his tenure at time \emph{0}
  depending on his relation to the past leader (or not); and,
\item
  the security of the regime when the leader holds office at time
  \emph{t}.
\end{enumerate}

\(\qquad\) At time \emph{0}, regime type matters. I use Archigo's data
set on leaders to obtain information of state leaders' names, their
tenures, the regime types when they hold power, manners of leader
transfers (Gleditsch and Chiozza 2015). The regime types are divided in
three cateogires: democracy, autocracy and no authority. They are
identified based on whether political leaders enter and exit in
political office based on explicit rules. V-Dem's Electoral Democracy
Index (2018) is also used to cross validate the regime type measurement.

\(\qquad\) An authoritarian leader is coded as secure at time 0 when he
is politically affiliated and from the same ruling coalition with his
previous leader. Otherwise, he is insecure at time 0. To measure
leader's relation to his past leader, I use Svolik (2012)`s
Authoritarian Spells, 1946 - 2008 data set that contains authoritarian
leaders' affiliation with their previous leaders. An authoritarian
leader is secure at time 0 when he is politically affiliated and from
the same ruling coalition with his previous leader. An authoritarian
leader is insecure at time 0 when a leader is in the opposition party of
the previous party, openly opposes the previous leader, or he is
unaffiliated (defined as not openly oppose or support) with the
preceding government (Svolik 2012, 43). In other cases, if he is in a
military regime or the regime with no authority, he is also defined as
insecure when he starts his tenure. Note that in democracies, a leader's
relation to his past is irrelevant; so there is no information of
leader's relation to his past leader in democracies. All the democratic
leaders are assumed as secure leaders before they serve in the office
(at time \emph{0}).

\(\qquad\) The security of the regime over time \emph{t} measures the
vulnerability of the regime to collapse in any given year. I use
political effectiveness score in the state fragility index to measure
the political secureness of a regime to capture the dimensions of
political opposition, citizen's confidence in the political process,
political violence of a regime and related political risks (Marshall and
Elzinga-Marshall 2017). The index ranges from 0 to 3, 0 means the most
secure, and 3 means insecure. I recoded the regime's security at time
\emph{0} as a binary variable: those with score 0 in political securenss
are coded secure, otherwise insecure. A worsening political
effectiveness score over a leader's tenure captures a political upheaval
that challenges the government authority. The wrosening score measures a
shock to leader's security that causes instability.

\(\qquad\) Here I develop two types of shocks to security. Type (1): a
leader who is presumably secure at time \emph{0} becomes insecure
immediately onward at time \emph{1}. Substantively, it means a
democratic leader or an authoritarian leader who politically affiliated
with his previous leaders starts his tenure in an unstable and contested
environment. The leaders may be highly constrained by the opposition and
deep internal political divisions. Type (2): a leader experiences a
shock to security during his tenure at time \emph{t}. The political
crisis captured in such a shock to security can be a result of threats
from the fractionalized ruling coalition, popular uprisings, or even the
actual use of force.

\(\qquad\) Here, I have developed the \emph{independent variable} as a
hypothesized treatment of shock to leader's security in the regime. It
is a binary variable. Once a leader experienced either or both types of
shocks to security, he is considered in the treatment group.

\(\qquad\) The \emph{dependent variable} is a binary one. If a leader in
the developing country has ever negotiated a South-North PTA with the
provisions of economic reforms during his tenure, then this event
happened, coded as 1, otherwise 0. Only the South-North preferential
trade agreements that contain a competition chapter or a competition
article is included. The information is collected on in DESTA data set
(Dür, Baccini, and Elsig 2014). The provision includes but is not
limited to the privatization of the state-owned firms and regulation on
monopolies and cartels. A leader may negotiate a couple of PTAs, but
only the first one will be only counted. In this data set, a majority of
the PTAs have been put into force, while some of them were still in the
negotiation process. Here I focus on the year leaders starts negotiation
rather than the year of signatory because there may be leadership change
between the years that a leader negotiates a PTA while another leader
ratifies it. I collected the information on the state's initiation of
PTA negotiations from 1995 to 2015.

\(\qquad\) The unit of observation in the data set is leader. The data
set covers 286 leaders clustered in 62 developing countries in the
period 1995 to 2015. Note that in this data set, at least one of the
leaders in these developing countries negotiated one PTA with the
provision of competition policy with a developed country at some point
during this period. The data set excludes leaders in the liberal
democracies (V-Dem's Electoral Democracy Index above 0.75) where the
rule of law and constraints on the executives are respected most of the
time. In a regime as such, leadership change is routine and
institutionalized. Therefore, there are lower risks for leaders losing
power and hence lower incentives for leaders to use binding trade
agreements for political survival. Note that under mature democratic
regimes, all but the most extreme form of opposition will be channeled
into the formal institutions of government, in which dissents of
opposition can hardly dampen the political stability or cause a shock to
security. In stable democracies, not only shocks to security rarely
happen but also leader turnovers are generally institutionalized and
peaceful. Leaders' reactions to such threats may also be
institutionalized. Therefore, I limit the scope of this study in only
autocratic and semi-democratic developing countries. Furthermore, it
removes leaders whose tenure is less than one year, in such case they
have no time to pursue any substantial policy changes given the short
amount of time in office.

\(\qquad\) I completely aware that the hypothetical treatment assignment
is not random in an observational study, so I include potential
\emph{confounding variables} to reduce ommited variable biases. The
covariates in this study include the qualities of democratic or
autocratic authority measured by V-Dem's Electoral Democracy Index
(Coppedge et al. 2018) because state leaders face different
institutional constraints when they initiate the trade agreement
negotiation. State leaders also face different risks of shock to
security in different regime types. The second covariate is leader's
tenure and the length of uninterrupted regime duration up to a leader
starts his tenure in Archigo data (Gleditsch and Chiozza 2015). A
leader's tenure is a confounder because the longer a leader holds
office, the probability of engaging in the PTA negotiation may be
higher, and the risks of being exposed to the shock to security is
higher.

\(\qquad\) Two additional control variables are included to take into
account the alternative hypothese. GDP per capita measured by the World
Bank (Bank 2018) is a binary variable to capture economic recession of
the developing countries between 1995 to 2015. In a period when a
country experiences poor economic performance, a leader in the
developing country perceives the potential economic benefits of the PTA
with the South can help the economy to recover. Also, this regime may be
more likely to experience negative shock of security when the economy
declines. Government is more likely to lose legitimacy in times of
economic difficulties; hence, it is more vulnerable to political risks.
Therefore, there is an alternative possibility that a leader may
negotiate a PTA driven by the economic benefits of the preferential
trade agreement.

\(\qquad\) Human rights conditions measured by Political Terror Scale
(Gib\(\-\)ney et al. 2017) indicate the human rights practices of the
regimes. A worsening Political Terror Scale measure an increased level
of violence by the state engaging in state-sanctioned killings, torture,
disappearances and political imprisonment. As illustrated above, a
dominant understanding of leader's response after the opposition poses
shock to security is to repress them. A worsening human rights
conditions is a measurement of leader's punishment tactic to the
opposition after he is challenged. If this competitive hypothesis holds,
leaders will engage in more human rights violations after they
experience the shock to security, a hypothetical treatment in this
study.

\begin{table}

\caption{\label{tab:xtable print pvalues}\label{tab:per} Percentages of Leaders being Treated and those Trated Having an Event}
\centering
\resizebox{\linewidth}{!}{
\begin{tabular}[t]{lrrr}
\toprule
  & All Regime Types & Democracies & Nondemocracies\\
\midrule
Percentages of leaders being treated (\%) & 32.52 & 31.10 & 36.36\\
Percentages of those treated negotiated a PTA (\%) & 44.09 & 35.38 & 64.29\\
\bottomrule
\end{tabular}}
\end{table}

\(\qquad\) Table \ref{tab:per} presents there are 93 events out of 286
total observations. Across regime types, 32.52\% of the leaders have
experienced shocks to security. Among leaders across regimes, 44.09\% of
those who have experienced a shock to security have chosen to negotiate
a PTA with a developed country at some point in their tenure. 26.42\% of
those who have not experienced a shock to security have negotiated a
South-North PTA. Now, let us look at the relative frequency distribution
divided by regime types. In non-democracies, 64.29\% of leaders have
experienced an insecure shock, while 31.1\% in democracies have
experienced one at some point during the time of their tenures. Among
leaders in non-democracies regimes, 36.73\% of those who have
experienced a shock to security negotiated a South-North PTA. In
contrast, only 31.1\% of those who have experienced a shock to security
in democracies negotiated one. Leaders in different regimes have
different patterns of negotiating PTAs. The modeling approach adopted in
this paper may allow us to account for such contextual factors.

\section{Model Analysis and
Discussion}\label{model-analysis-and-discussion}

\subsection{Random Intercept Models}\label{random-intercept-models}

\(\qquad\) The data set in this paper exhibit a nested structure. The
level-one units are state leaders, with a sample size of 286. The
level-two units are 62 developing countries. Because of missing data,
the sample size of level-one drops to 218, the sample size of level-two
drops to 49. Table \ref{tab:summary_s} summarizes the descriptive
statistics. If the models I use ignore the fact that individual leaders
are clustered within different countries, I may run the risks of getting
downwardly biased standard errors, which to inflated Type I errors. We
may often use clustered standard errors to correct for the clustering
data structure. Although it may be a solution of solving statistical
problems, we may miss the opporunities to explore the theoretical
questions of multilevel data. This research note will adopt a
hierarchical linear modeling tehnique to answer the following question:
Do individual leaders' shocks to security lead to leaders' decision of
PTA negotiations? Results are shown in Table \ref{tab:re}.

\rowcolors{2}{gray!6}{white}

\begin{table}[!h]

\caption{\label{tab:xtable print summary_s}\label{tab:summary_s} Descriptive Statistics}
\centering
\begin{tabular}[t]{llllll}
\hiderowcolors
\toprule
  & N & Mean & SD & Min. & Max.\\
\midrule
\showrowcolors
\addlinespace[0.3em]
\multicolumn{6}{l}{\textbf{Individual-Level}}\\
\hspace{1em}PTA Negotiation & 218 & 0.32 & 0.47 & 0 & 1\\
\hspace{1em}Shock to security & 218 & 0.33 & 0.47 & 0 & 1\\
\hspace{1em}Mean tenure of Leader & 218 & 6.66 & 6.51 & 2 & 45\\
\hspace{1em}Mean human rights violation & 218 & 2.87 & 0.98 & 1 & 5\\
\addlinespace[0.3em]
\multicolumn{6}{l}{\textbf{Group-Level}}\\
\hspace{1em}GDP per capita (log) & 49 & 8.41 & 0.92 & 6.23 & 11.02\\
\hspace{1em}Level of democracy & 49 & 0.39 & 0.19 & 0.04 & 0.74\\
\hspace{1em}Age of democracy & 49 & 15.31 & 13.04 & 0 & 42\\
\hspace{1em}Age of autocracy & 49 & 9.22 & 14.43 & 0 & 44\\
\hspace{1em}Economic recession & 49 & 0.8 & 0.41 & 0 & 1\\
\bottomrule
\end{tabular}
\end{table}

\rowcolors{2}{white}{white}

\begin{landscape}\rowcolors{2}{gray!6}{white}
\begin{table}[!h]

\caption{\label{tab:xtable print}\label{tab:re} Four Random Intercept Models of Explaining PTA Negotiation: Estimates, P-values and Variance Components}
\centering
\begin{tabular}[t]{llllllll}
\hiderowcolors
\toprule
\multicolumn{1}{c}{ } & \multicolumn{7}{c}{PTA Negotiation Rate (\%)} \\
\cmidrule(l{2pt}r{2pt}){2-8}
\multicolumn{1}{c}{ } & \multicolumn{1}{c}{Model 0} & \multicolumn{2}{c}{Model 1.1} & \multicolumn{2}{c}{Model 1.2} & \multicolumn{2}{c}{Model 1.3} \\
\cmidrule(l{2pt}r{2pt}){2-2} \cmidrule(l{2pt}r{2pt}){3-4} \cmidrule(l{2pt}r{2pt}){5-6} \cmidrule(l{2pt}r{2pt}){7-8}
  &   & Estimates & Pr(>|t|) & Estimates & Pr(>|t|) & Estimates & Pr(>|t|)\\
\midrule
\showrowcolors
\addlinespace[0.3em]
\multicolumn{8}{l}{\textbf{Individual-Level}}\\
\hspace{1em}Shock to security & - & - & - & - & - & 0.12 & 0.047\\
\hspace{1em}Tenure of Leader & - & - & - & 0.02 & 0.017 & 0.19 & 0.017\\
\hspace{1em}Mean human rights violation & - & - & - & - & - & -0.08 & 0.06\\
\addlinespace[0.3em]
\multicolumn{8}{l}{\textbf{Group-Level}}\\
\hspace{1em}Level of democracy & - & -0.43 & 0.09 & -0.64 & 0.03 & -0.58 & 0.004\\
\hspace{1em}Age of democracy & - & 0.00 & 0.60 & - & - & - & -\\
\hspace{1em}Age of autocracy & - & 0.00 & 0.14 & - & - & - & -\\
\hspace{1em}GDP per capita (log) & - & - & - & 0.06 & 0.188 & 0.07 & 0.141\\
\hspace{1em}Economic recession & - & - & - & -0.09 & 0.395 & -0.09 & 0.39\\
\addlinespace[0.3em]
\multicolumn{8}{l}{\textbf{Random Effects}}\\
\hspace{1em}Level-2 variance (between state) & 0.012 & 0.011 & - & 0.016 & - & 0.014 & -\\
\hspace{1em}Level-1 variance (within states) & 0.208 & 0.2 & - & 0.191 & - & 0.191 & -\\
\hspace{1em}Variation explained at level 1 & 94.71\% & 3.62\% & - & 7.03\% & - & 8.05\% & -\\
\hspace{1em}Variation explained at level 2 & 5.3\% & 3.27\% & - & - & - & - & -\\
\bottomrule
\end{tabular}
\end{table}
\rowcolors{2}{white}{white}
\end{landscape}

\(\qquad\) The first step in building multilevel models is to decompose
the variance in the dependent variable into two levels, the individual
leader and state levels. This is equivalent to running a random
intercept model with no explanatory variables, where I do not specify
any predictors of PTA negotiation. It is often called an empty model. In
decomposing the variance of dependent variable in this way, I can find
the proportion of variance in the dependent variable attributable to
between-cluster differences and the proportion attributable to
within-cluster variability. I find that 5.3\% of the variance in PTA
negotiation explained at the country level, with the remaining 94.7\%
came from between-individual variation. Some variations at the
state-level allows me to explore these contxtual variations in the
multilevel model setting.

\(\qquad\) The second step is to build the multilevel models, starting
from simple to a full and comlicated one. Only the level-two variables
are added in the random intercept model. Three regime quality related
state-level variables are added to see the effects of these variables on
PTA negotiation. These three variables are the \emph{age of democracy},
the \emph{age of autocracy}, and the \emph{level of democracy}. The
second step is to include state-level covariates to see if countries
with longer history of democracy or autocracy have positive association
with the negotiate rate of preferential trade agreements. The reason
lies in countries tend to form long-term relationship with trade
partners and avoid transitioning and unstable countries. Also, when the
developing countries' level of democracy increases, I expect there are
more trade cooperation between these countries and developed countries
that are most stable democracies. Reimge similarities may reward more
trade opportunities among partners. Table \ref{tab:per} Model 1.1 shows
the results. The ages of regimes show no substantive effects of PTA
negotiation. The level of democracy shows the oppositie direction as I
expected. All of these three variables have no statistical signifiance.
This model has explained 3.27\% of the level-2 variation captrued in the
empty model.

\(\qquad\) Model 1.2 keeps one of the level-2 variables used in Model
1.1, \emph{level of democracy}, and adds in the variables that measure
the economic conditions of states: \emph{GDP per capita (logged)} and
\emph{economic recession}. A level-1 variable, the \emph{mean tenure of
leader} is also added in the model. This variable is centered within the
groups. Note that the estimates of the two models for the three level-2
variables are the same, so I only show the result of the slightly
complicated model with three level-2 variables and one level-1 variable.
We can see level of democracy has statistical signifiance; yet, the
estimate presents a negative relation between developing countries'
level of democracy and PTA negotiation rates. The range of the level of
democracy is from 0 to 1; if a regime's level of democracy increases
from 0 to 1, suggesting an extreme transition from the most autocratic
to the most democratic regime, the PTA negotiation rate goes down by
64\%. More realistically, if a regime's level of democracy improves by
0.1 point in V-Dem, the probability of the PTA negotiation goes down by
6.4\%.

\(\qquad\) I then add more level-1 variables, the \emph{mean human
rights conditions} and the main independent variable in this paper, the
\emph{shock to security}. The estimates in the random intercept models
with and without the \emph{shock to security} variable are similar, so I
show the full model in Model 1.3. The variables \emph{mean tenure of
leader}, \emph{GDP per capita(logged)} and \emph{shock to security} have
statisctical meanings, and the \emph{mean human rights violation} is
slightly above the threshold of rejecting the null hypothesis of no
statistical significance (p-value 0.05). This full model only explains
8\% of the level-1 variation identified in the empty model, which still
leaves a large proportion of variation unexplained. Substantively, the
full model shows holding other leader-related factors, political and
economic conditions constant, when there is a political shock to
security, the leader will increase the odds of PTA negotiation by 12\%.
When a leader holds office one year longer than the mean tenure years of
leaders of that specific state, the leader will increase the probability
of PTA negotiation by 2\% when other factors are held constant.

\subsection{Brief Discussion on the Choice of Random Intercept and Slope
Model}\label{brief-discussion-on-the-choice-of-random-intercept-and-slope-model}

To explore whether the effects of level-1 factors are conditioned by
group-level factors, I continue to build random intercept and slope
models. In Model 2.1, I allow the effect of shock to security varies
across states with high and low levels of democracy. In Model 2.2, I
extend Model 2.1 to account for the effect of human rights condition
varies across states as well.

\(\qquad\)The mixed model 2.1 is:
\[ Y = \gamma_{00} + \gamma_{01}*(Log(GDP))_j + \gamma_{02}*(Democ)_j + \gamma_{03}*Recession_j + \gamma_{10}*(Shock) + \gamma_{11}*(Democ_{ij} - \overline{Democ_{ij}}) + \beta_{2j}*(HumanRights)_{ij} + \beta_{3j}*(Tenure_{ij} - \overline{Tenure_{ij}) + \miu_{oj} + \miu_{1j}(Shock) + R_{oj} + \epsilon \]

\section{Preliminary Conclusion}\label{preliminary-conclusion}

Why and when will leaders in developing country negotiate South-North
Preferential Trade Agreements? The answer is simple and intuitive: after
leaders in developing country experience a negative shock that creates
political instability, they are more likely to negotiate a PTA with the
provision of economic reforms with the expectation to cut off the power
sources of the disloyal opposition. Economic reform, hence, is a tactic
for leader to punish the opposition and a toolkit to consolidate power
in the regime. I have used original data on measuring security of
leaders in developing countries between 1995 and 2015 to examine whether
developing country leaders are more likely to negotiate a PTA in their
tenures when they experience a political shock. I argue leaders in
developing country approach the major liberal trade powers to negotiate
a PTA with the provision of economic reform, so they have chosen the
deepest PTAs by design. This is a strategic choice that leader makes
with an objective to punish the opposition who defect from the promised
support and pose the threatening demands.

\newpage

\section*{References}\label{references}
\addcontentsline{toc}{section}{References}

\hypertarget{refs}{}
\hypertarget{ref-bacciniInternationalInstitutionsDomestic2014}{}
Baccini, Leonardo, and Johannes Urpelainen. 2014. ``International
Institutions and Domestic Politics: Can Preferential Trading Agreements
Help Leaders Promote Economic Reform?'' \emph{The Journal of Politics}
76 (1): 195--214.
doi:\href{https://doi.org/10.1017/S0022381613001278}{10.1017/S0022381613001278}.

\hypertarget{ref-bagwellWillPreferentialAgreements1998}{}
Bagwell, Kyle, and Robert W. Staiger. 1998. ``Will Preferential
Agreements Undermine the Multilateral Trading System?'' \emph{The
Economic Journal} 108 (449): 1162--82.
doi:\href{https://doi.org/10.1111/1468-0297.00336}{10.1111/1468-0297.00336}.

\hypertarget{ref-baldwinDominoTheoryRegionalism2012}{}
Baldwin, Richard E. 2012. ``A Domino Theory of Regionalism.'' In
\emph{The Economics of Free Trade. Volume 2.}, edited by Gary Hufbauer
and Kati Suominen, 416--39. Elgar Research Collection. The International
Library of Critical Writings in Economics, vol. 262. Cheltenham, U.K.
and Northampton, Mass.: Elgar.

\hypertarget{ref-world_bank_world_2018}{}
Bank, World. 2018. ``World Development Indicators.''
\url{https://data.worldbank.org/indicator/NY.GDP.PCAP.CD}.

\hypertarget{ref-coppedge_v-dem_2018}{}
Coppedge, Michael, John Gerring, Carl Henrik Knutsen, Staffan I.
Lindberg, Svend-Erik Skaaning, Jan Teorell, David Altman, et al. 2018.
``V-Dem Country-Year Dataset 2018.'' Varieties of Democracy (V-Dem)
Project.
doi:\href{https://doi.org/10.23696/vdemcy18}{10.23696/vdemcy18}.

\hypertarget{ref-durDesignInternationalTrade2014}{}
Dür, Andreas, Leonardo Baccini, and Manfred Elsig. 2014. ``The Design of
International Trade Agreements: Introducing a New Dataset.'' \emph{The
Review of International Organizations} 9 (3): 353--75.
doi:\href{https://doi.org/10.1007/s11558-013-9179-8}{10.1007/s11558-013-9179-8}.

\hypertarget{ref-gibney_politicterror_2017}{}
Gib\(\-\)ney, Mark, Linda Cornett, Reed Wood, Peter Hasch\(\-\)ke,
Daniel Arnon, and Attilio Pisanò. 2017. ``The Polit\(\-\)ic\(\-\)al
Ter\(\-\)ror Scale 1976-2016.''
\url{ht/\%00adtp://www.polit/\%00adic/\%00adal/\%00adter/\%00adrorscale.org.}

\hypertarget{ref-gleditsch_archigos_2015}{}
Gleditsch, Kristian Skrede, and Giacomo Chiozza. 2015. ``Archigos
Versopm 4.1.''
\url{http://www.rochester.edu/college/faculty/hgoemans/data.htm}.

\hypertarget{ref-mansfieldMultilateralDeterminantsRegionalism2003}{}
Mansfield, Edward D., and Eric Reinhardt. 2003. ``Multilateral
Determinants of Regionalism: The Effects of GATT/WTO on the Formation of
Preferential Trading Arrangements.'' \emph{International Organization}
57 (4): 829--62.

\hypertarget{ref-marshall_global_2017}{}
Marshall, Monty G., and Gabrielle Elzinga-Marshall. 2017. ``Global
Report 2017: Conflict, Governance and State Fragility.'' Center for
Systemic Peace. \url{http://fundforpeace.org/fsi/indicators/}.

\hypertarget{ref-rodrikRushFreeTrade1992}{}
Rodrik, Dani. 1992. ``The Rush to Free Trade in the Developing World:
Why so Late? Why Now? Will It Last?'' w3947. Cambridge, MA: National
Bureau of Economic Research.
doi:\href{https://doi.org/10.3386/w3947}{10.3386/w3947}.

\hypertarget{ref-svolik_politics_2012}{}
Svolik, Milan. 2012. \emph{The Politics of Authoritarian Rule.} New
York: Cambridge University Pres.


\end{document}
